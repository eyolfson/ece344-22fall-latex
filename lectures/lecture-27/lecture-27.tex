\documentclass[aspectratio=169]{beamer}

\usepackage{ccicons}
\usepackage{fontspec}
\usepackage{listings}
\usepackage{tikz}
\usepackage{svg}

\definecolor{uclablue}{RGB}{39,116,174}
\definecolor{uclagold}{RGB}{255,179,0}

\definecolor{ubcorange}{RGB}{158, 66, 37}

\definecolor{cugold}{RGB}{207, 184, 124}
\definecolor{cudarkgray}{RGB}{86, 90, 92}

\definecolor{solarizedred}{RGB}{220, 50, 47}
\definecolor{solarizedblue}{RGB}{38, 139, 210}
\definecolor{solarizedgreen}{RGB}{133, 153, 0}
\definecolor{solarizedpurple}{RGB}{108, 113, 196}
\definecolor{solarizedmagenta}{RGB}{211, 54, 130}

\definecolor{pantone655}{RGB}{0, 42, 92}
\definecolor{pantone7453}{RGB}{123, 164, 217}
\definecolor{pantone633}{RGB}{0, 139, 176}
\definecolor{pantone7492}{RGB}{218, 229, 205}

\colorlet{primarycolor}{pantone655}
\colorlet{secondarycolor}{pantone7453}


\usetikzlibrary{
  arrows,
  arrows.meta,
  automata,
  backgrounds,
  calc,
  chains,
  decorations.pathreplacing,
  fit,
  intersections,
  matrix,
  overlay-beamer-styles,
  positioning,
  shapes,
  shapes.multipart,
  tikzmark,
}
\usetikzmarklibrary{listings}

\hypersetup{
  colorlinks=true,
  urlcolor=cudarkgray,
}

\setbeamercolor{frametitle}{fg=primarycolor}
\setbeamercolor{structure}{fg=primarycolor}
\setbeamercolor{enumerate item}{fg=black}
\setbeamercolor{itemize item}{fg=black}
\setbeamercolor{itemize subitem}{fg=black}

\setbeamersize{text margin left=26.6mm}
\addtolength{\headsep}{2mm}

\setbeamertemplate{navigation symbols}{}
\setbeamertemplate{headline}{}
\setbeamertemplate{footline}{}
\setbeamertemplate{itemize item}{\color{black}}
\setbeamertemplate{itemize items}[circle]

\setbeamertemplate{footline}{
  \begin{tikzpicture}[remember picture,
                      overlay,
                      shift={(current page.south west)}]
    \node [black!50, inner sep=2mm, anchor=south east]
          at (current page.south east) {\footnotesize \insertframenumber};
  \end{tikzpicture}
}

\setsansfont{Inter}[Scale=MatchLowercase]
\setmonofont{Hack}[Scale=MatchLowercase]

\makeatletter
\newcommand\version[1]{\renewcommand\@version{#1}}
\newcommand\@version{}
\def\insertversion{\@version}

\newcommand\lecturenumber[1]{\renewcommand\@lecturenumber{#1}}
\newcommand\@lecturenumber{}
\def\insertlecturenumber{\@lecturenumber}
\makeatother

\setbeamertemplate{title page}
{
  \begin{tikzpicture}[remember picture,
                      overlay,
                      shift={(current page.south west)},
                      background rectangle/.style={fill=pantone655},
                      show background rectangle]
    \node [anchor=west, align=left, inner sep=0, text=white]
          (lecturenumber) at (\paperwidth / 6, \paperheight * 3 / 4)
          {\Large Lecture \insertlecturenumber};
    \node [inner sep=0, align=left, text=white, node distance=0,
          above left=of lecturenumber, anchor=south west, yshift=2mm]
          {\Large ECE 344: Operating Systems};
    \node (title) [inner sep=0, anchor=west, align=left, text=white,
                   text width=30em]
          at (\paperwidth / 6, \paperheight / 2)
          {{\bfseries \Huge \inserttitle{}}};
    \node [inner sep=0, align=right, text=white, node distance=0,
          below right=of title, anchor=north east, yshift=-1mm]
          {{\footnotesize \ttfamily \insertversion}};
    \node [inner sep=0, text=white, align=left, anchor=west]
          (author) at (\paperwidth / 6, \paperheight / 4)
          {\insertauthor};
    \node [text=white, inner sep=0, align=left, node distance=0,
           below left=of author, anchor=north west, yshift=-2mm]
          {\insertdate};
    \node [align=right, anchor=south east, inner sep=2mm, text=white]
          (license) at (\paperwidth, 0)
          {\footnotesize This  work is licensed under a
           \href{http://creativecommons.org/licenses/by-sa/4.0/}
                {\color{pantone7453} Creative Commons Attribution-ShareAlike 4.0
                 International License}};
    \node [text=white, inner sep=0, align=right, node distance=0,
           above right=of license, anchor=south east, xshift=-2mm]
          {\Large \ccbysa};
  \end{tikzpicture}
}

\tikzset{
  >=Straight Barb[],
  shorten >=1pt,
  initial text=,
}

\lstset{
  basicstyle=\footnotesize\ttfamily,
  language=C,
  escapechar=@,
  commentstyle=\color{black!50},
}


\lecturenumber{27}
\title{Quiz 3 Review}
\version{1.0.0}
\author{Jon Eyolfson}
\date{November 21, 2022}

\begin{document}
  \begin{frame}[plain, noframenumbering]
    \titlepage
  \end{frame}

  \begin{frame}{The Quiz Covers 4 General Topics}

    \begin{itemize}
      \item Scheduling
      \item Page Replacement
      \item Page Tables and TLB
      \item Virtual Memory
    \end{itemize}

    \vspace{2em}

    \textit{No sockets}
  \end{frame}

  \begin{frame}
    \frametitle{Scheduling Involves Trade-Offs}

    We looked at few different algorithms:
    \begin{itemize}
      \item First Come First Served (FCFS) is the most basic scheduling algorithm
      \item Shortest Job First (SJF) is a tweak that reduces waiting time
      \item Shortest Remaining Time First (SRTF) uses SJF ideas with preemptions
      \item SRTF optimizes lowest waiting time (or turnaround time)
      \item Round-robin (RR) optimizes fairness and response time
    \end{itemize}
  \end{frame}
  
  \begin{frame}
    \frametitle{Scheduling Gets Even More Complex}

    There are more solutions, and more issues:
    \begin{itemize}
      \item Introducing priority also introduces priority inversion
      \begin{itemize}
        \item We say dynamic priority scheduling
      \end{itemize}
      \item Some processes need good interactivity, others not so much
      \item Multiprocessors may require per-CPU queues
      \item Real-time requires predictability
      \item Completely Fair Scheduler (CFS) tries to model the ideal fairness
    \end{itemize}
  \end{frame}

  \begin{frame}
    \frametitle{Page Tables Translate Virtual to Physical Addresses}

    The MMU is the hardware that uses page tables, which may:
    \begin{itemize}
      \item Be a single large table (wasteful, even for 32-bit machines)
      \item Be a multi-level to save space for sparse allocations
      \item Use the kernel allocate pages from a free list
      \item Use a TLB to speed up memory accesses
    \end{itemize}
  \end{frame}

  \begin{frame}
    \frametitle{Page Replacement Algorithms Aim to Reduce Page Faults}

    We saw the following:
    \begin{itemize}
      \item Optimal (good for comparison but not realistic)
      \item Random (actually works surprisingly well, avoids the worst case)
      \item FIFO (easy to implement but Bélády's anomaly)
      \item LRU (gets close to optimal but expensive to implement)
    \end{itemize}

    \vspace{2em}

    The pages go to a swap file on disk
  \end{frame}
  
  \begin{frame}{A Bitmap Is What a Slab Allocator Uses}

    It's a contiguous block of memory that tracks slots (1 bit each)

    \vspace{2em}

    Assume we have a bitmap that's 512 bytes large

    \hspace{2em} That's 4096 bits

    \vspace{2em}

    The bitmap could track 4096 slots (e.g. bit 500 tracks slot 500)
  \end{frame}

  \begin{frame}{The Coremap Data Structure Manages the Physical Pages (Frames)}

    It keeps track of what processes can access which frames

    \hspace{2em} Recall: two processes could map different virtual pages to the
                 same frame

    \vspace{2em}

    For each frame it would track what processes (pid) have access to it

    \vspace{2em}

    It would be able to free the frame if no processes have it mapped
  \end{frame}

  \begin{frame}{Rough Memory Heirarchy Latencies}

    CPU cache: order of nanoseconds ($\mathsf{1\ ns}$)

    \vspace{2em}

    Memory: 100s of nanoseconds, or single digit microseconds
            ($\mathsf{1\ \mu s}$)

    \vspace{2em}

    SSD: double digit microseconds ($\mathsf{10\ \mu s}$)

    \vspace{2em}

    Disk (Hard Drive): possibly up to milliseconds ($\mathsf{1\ ms}$)
  \end{frame}

  \begin{frame}{Spatial vs Temporal Locality}

    Spatial locality means accesses are close to each other in memory

    (accessed contiguously or almost)

    \vspace{2em}

    Temporal locality means accesses are close to each other in a small time
    duration (accessed together or almost)
  \end{frame}
  
  \begin{frame}{Thrashing Is When There is a High Number of Page Faults}

    This occurs when processes cannot have their working sets of memory loaded

    \hspace{2em} A working set is an amount of memory a process requires in a
                 time interval

    \vspace{2em}

    Processes constantly have to evict pages they are currently using

    \vspace{2em}

    This causes the system to be very slow and almost unresponsive

    \vspace{2em}

    If the page fault frequency (PFF) is too high, the process
    
    doesn't have enough memory for its working set
  \end{frame}
  
  \begin{frame}{Demand Paging}

    Mark a page in the page table as invalid

    \hspace{2em} This generates a page fault on the first access

    \vspace{2em}

    The kernel can then allocate a page and update the page table with a valid
    entry

    \vspace{2em}

    Page fault may happen because a process accesses an unmapped
    virtual address

    \hspace{2em} Instead send a segfault to the process

    \vspace{2em}

    Note, the alternative is to load everything into memory when the program
    starts
  \end{frame}

  \begin{frame}{Copy-on-Write (COW) is a Useful Technique}

    Similar to data races, we can share data as long as there's only reads

    \vspace{2em}

    If two processes are sharing the same frame, and either could write (e.g.
    right after a fork)

    \hspace{2em} Only copy the frame after one process modifies its page

    \vspace{2em}

    We can also do other optimizations with read-only frames
    
    (e.g. one representing the executable code)

    \hspace{2em} Instead of evicting page to disk, just re-read again when you
                 need it
  \end{frame}
\end{document}
