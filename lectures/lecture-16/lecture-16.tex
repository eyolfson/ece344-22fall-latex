\documentclass[aspectratio=169]{beamer}

\usepackage{ccicons}
\usepackage{fontspec}
\usepackage{listings}
\usepackage{tikz}
\usepackage{svg}

\definecolor{uclablue}{RGB}{39,116,174}
\definecolor{uclagold}{RGB}{255,179,0}

\definecolor{ubcorange}{RGB}{158, 66, 37}

\definecolor{cugold}{RGB}{207, 184, 124}
\definecolor{cudarkgray}{RGB}{86, 90, 92}

\definecolor{solarizedred}{RGB}{220, 50, 47}
\definecolor{solarizedblue}{RGB}{38, 139, 210}
\definecolor{solarizedgreen}{RGB}{133, 153, 0}
\definecolor{solarizedpurple}{RGB}{108, 113, 196}
\definecolor{solarizedmagenta}{RGB}{211, 54, 130}

\definecolor{pantone655}{RGB}{0, 42, 92}
\definecolor{pantone7453}{RGB}{123, 164, 217}
\definecolor{pantone633}{RGB}{0, 139, 176}
\definecolor{pantone7492}{RGB}{218, 229, 205}

\colorlet{primarycolor}{pantone655}
\colorlet{secondarycolor}{pantone7453}


\usetikzlibrary{
  arrows,
  arrows.meta,
  automata,
  backgrounds,
  calc,
  chains,
  decorations.pathreplacing,
  fit,
  intersections,
  matrix,
  overlay-beamer-styles,
  positioning,
  shapes,
  shapes.multipart,
  tikzmark,
}
\usetikzmarklibrary{listings}

\hypersetup{
  colorlinks=true,
  urlcolor=cudarkgray,
}

\setbeamercolor{frametitle}{fg=primarycolor}
\setbeamercolor{structure}{fg=primarycolor}
\setbeamercolor{enumerate item}{fg=black}
\setbeamercolor{itemize item}{fg=black}
\setbeamercolor{itemize subitem}{fg=black}

\setbeamersize{text margin left=26.6mm}
\addtolength{\headsep}{2mm}

\setbeamertemplate{navigation symbols}{}
\setbeamertemplate{headline}{}
\setbeamertemplate{footline}{}
\setbeamertemplate{itemize item}{\color{black}}
\setbeamertemplate{itemize items}[circle]

\setbeamertemplate{footline}{
  \begin{tikzpicture}[remember picture,
                      overlay,
                      shift={(current page.south west)}]
    \node [black!50, inner sep=2mm, anchor=south east]
          at (current page.south east) {\footnotesize \insertframenumber};
  \end{tikzpicture}
}

\setsansfont{Inter}[Scale=MatchLowercase]
\setmonofont{Hack}[Scale=MatchLowercase]

\makeatletter
\newcommand\version[1]{\renewcommand\@version{#1}}
\newcommand\@version{}
\def\insertversion{\@version}

\newcommand\lecturenumber[1]{\renewcommand\@lecturenumber{#1}}
\newcommand\@lecturenumber{}
\def\insertlecturenumber{\@lecturenumber}
\makeatother

\setbeamertemplate{title page}
{
  \begin{tikzpicture}[remember picture,
                      overlay,
                      shift={(current page.south west)},
                      background rectangle/.style={fill=pantone655},
                      show background rectangle]
    \node [anchor=west, align=left, inner sep=0, text=white]
          (lecturenumber) at (\paperwidth / 6, \paperheight * 3 / 4)
          {\Large Lecture \insertlecturenumber};
    \node [inner sep=0, align=left, text=white, node distance=0,
          above left=of lecturenumber, anchor=south west, yshift=2mm]
          {\Large ECE 344: Operating Systems};
    \node (title) [inner sep=0, anchor=west, align=left, text=white,
                   text width=30em]
          at (\paperwidth / 6, \paperheight / 2)
          {{\bfseries \Huge \inserttitle{}}};
    \node [inner sep=0, align=right, text=white, node distance=0,
          below right=of title, anchor=north east, yshift=-1mm]
          {{\footnotesize \ttfamily \insertversion}};
    \node [inner sep=0, text=white, align=left, anchor=west]
          (author) at (\paperwidth / 6, \paperheight / 4)
          {\insertauthor};
    \node [text=white, inner sep=0, align=left, node distance=0,
           below left=of author, anchor=north west, yshift=-2mm]
          {\insertdate};
    \node [align=right, anchor=south east, inner sep=2mm, text=white]
          (license) at (\paperwidth, 0)
          {\footnotesize This  work is licensed under a
           \href{http://creativecommons.org/licenses/by-sa/4.0/}
                {\color{pantone7453} Creative Commons Attribution-ShareAlike 4.0
                 International License}};
    \node [text=white, inner sep=0, align=right, node distance=0,
           above right=of license, anchor=south east, xshift=-2mm]
          {\Large \ccbysa};
  \end{tikzpicture}
}

\tikzset{
  >=Straight Barb[],
  shorten >=1pt,
  initial text=,
}

\lstset{
  basicstyle=\footnotesize\ttfamily,
  language=C,
  escapechar=@,
  commentstyle=\color{black!50},
}


\lecturenumber{16}
\title{Lab 4 Primer}
\version{1.0.0}
\author{Jon Eyolfson}
\date{October 17, 2022}

\begin{document}
  \begin{frame}[plain, noframenumbering]
    \titlepage
  \end{frame}

  \begin{frame}
    \frametitle{We Want to Send and Recieve Data From a Process}

    \begin{enumerate}
      \item Create a new process that launches the command line argument
      \item Send the string \texttt{Testing\textbackslash n} to that process
      \item Receive any data it writes to standard output
    \end{enumerate}
  \end{frame}

  \begin{frame}
    \frametitle{Our First New API --- \texttt{pipe}}

    \lstinline!int pipe(pipefd[2]);!

    \vspace{2em}

    Returns \texttt{0} on success, and \texttt{-1} on failure
    (and sets \texttt{errno})

    \vspace{2em}

    \texttt{pipe} forms a one-way communication channel using two file
    descriptors

    \hspace{2em} \texttt{pipefd[0]} is the \texttt{read} end of the pipe

    \hspace{2em} \texttt{pipefd[1]} is the \texttt{write} end of the pipe

    \vspace{2em}

    You can think of it as a kernel managed buffer
    
    \hspace{2em} Any data written to one end can be read on the other end
  \end{frame}

  \begin{frame}
    \frametitle{A More Convenient API -- \texttt{execlp}}

    \lstinline!int execlp(const char *file, const char *arg /*..., (char *)
                          NULL */);!

    \vspace{2em}

    Does not return on success, and \texttt{-1} on failure
    (and sets \texttt{errno})

    \vspace{2em}

    \texttt{execlp} will let you skip using string arrays (using C varargs),
    
    and it will also search for executables using the \texttt{PATH} environment
    variable
  \end{frame}

  \begin{frame}
    \frametitle{Our Next API --- \texttt{close}}

    \lstinline!int close(int fd);!

    \vspace{2em}

    Returns \texttt{0} on success, and \texttt{-1} on failure
    (and sets \texttt{errno})

    \vspace{2em}

    Closes the file descriptor for the process, no longer usable

    \vspace{2em}

    This frees up the file desciptor (recall, it's just a number) to be reused
  \end{frame}

  \begin{frame}
    \frametitle{Our Final APIs --- \texttt{dup} and \texttt{dup2}}

    \lstinline!int dup(int oldfd);!

    \lstinline!int dup2(int oldfd, int newfd);!

    \vspace{2em}

    Returns a new file descriptor on success, and \texttt{-1} on failure
    (and sets \texttt{errno})

    \vspace{2em}

    Copies the file descriptor so \texttt{oldfd} and \texttt{newfd} refer to
    the same thing

    \vspace{2em}

    For \texttt{dup} it'll return the lowest file descriptor

    \vspace{2em}

    For \texttt{dup2} it'll atomically close the \texttt{newfd} argument
    (if open),
    
    and then make \texttt{newfd} refer to the same thing
  \end{frame}

  \begin{frame}[fragile]
    \frametitle{Coding Example}

    Done live!

    \vspace{2em}

    You can find the template in \texttt{lecture-16} in the \texttt{examples}
    repository

    \vspace{2em}

    To compile it, run the following commands:

    \begin{lstlisting}[language=bash, xleftmargin=2em]
cd lecture-16 # if not already there
mkdir build
cd build
cmake ..
cmake --build . # or make
    \end{lstlisting}

    \vspace{2em}

    Run the program using: \lstinline!./subprocess <program>!
  \end{frame}

  \begin{frame}[fragile]
    \frametitle{Running with \texttt{cat} May Cause Problems}

    Run the program with the following arguments:

    \vspace{2em}

    \begin{lstlisting}[language=, xleftmargin=2em]
./subprocess uname
./subprocess cat
    \end{lstlisting}

    \vspace{2em}

    You have to be careful with the file descriptors!

    \vspace{2em}

    Why might \texttt{cat} not exit when using pipes?
  \end{frame}
\end{document}
