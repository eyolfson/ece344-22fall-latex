\documentclass[aspectratio=169]{beamer}

\usepackage{ccicons}
\usepackage{fontspec}
\usepackage{listings}
\usepackage{tikz}
\usepackage{svg}

\definecolor{uclablue}{RGB}{39,116,174}
\definecolor{uclagold}{RGB}{255,179,0}

\definecolor{ubcorange}{RGB}{158, 66, 37}

\definecolor{cugold}{RGB}{207, 184, 124}
\definecolor{cudarkgray}{RGB}{86, 90, 92}

\definecolor{solarizedred}{RGB}{220, 50, 47}
\definecolor{solarizedblue}{RGB}{38, 139, 210}
\definecolor{solarizedgreen}{RGB}{133, 153, 0}
\definecolor{solarizedpurple}{RGB}{108, 113, 196}
\definecolor{solarizedmagenta}{RGB}{211, 54, 130}

\definecolor{pantone655}{RGB}{0, 42, 92}
\definecolor{pantone7453}{RGB}{123, 164, 217}
\definecolor{pantone633}{RGB}{0, 139, 176}
\definecolor{pantone7492}{RGB}{218, 229, 205}

\colorlet{primarycolor}{pantone655}
\colorlet{secondarycolor}{pantone7453}


\usetikzlibrary{
  arrows,
  arrows.meta,
  automata,
  backgrounds,
  calc,
  chains,
  decorations.pathreplacing,
  fit,
  intersections,
  matrix,
  overlay-beamer-styles,
  positioning,
  shapes,
  shapes.multipart,
  tikzmark,
}
\usetikzmarklibrary{listings}

\hypersetup{
  colorlinks=true,
  urlcolor=cudarkgray,
}

\setbeamercolor{frametitle}{fg=primarycolor}
\setbeamercolor{structure}{fg=primarycolor}
\setbeamercolor{enumerate item}{fg=black}
\setbeamercolor{itemize item}{fg=black}
\setbeamercolor{itemize subitem}{fg=black}

\setbeamersize{text margin left=26.6mm}
\addtolength{\headsep}{2mm}

\setbeamertemplate{navigation symbols}{}
\setbeamertemplate{headline}{}
\setbeamertemplate{footline}{}
\setbeamertemplate{itemize item}{\color{black}}
\setbeamertemplate{itemize items}[circle]

\setbeamertemplate{footline}{
  \begin{tikzpicture}[remember picture,
                      overlay,
                      shift={(current page.south west)}]
    \node [black!50, inner sep=2mm, anchor=south east]
          at (current page.south east) {\footnotesize \insertframenumber};
  \end{tikzpicture}
}

\setsansfont{Inter}[Scale=MatchLowercase]
\setmonofont{Hack}[Scale=MatchLowercase]

\makeatletter
\newcommand\version[1]{\renewcommand\@version{#1}}
\newcommand\@version{}
\def\insertversion{\@version}

\newcommand\lecturenumber[1]{\renewcommand\@lecturenumber{#1}}
\newcommand\@lecturenumber{}
\def\insertlecturenumber{\@lecturenumber}
\makeatother

\setbeamertemplate{title page}
{
  \begin{tikzpicture}[remember picture,
                      overlay,
                      shift={(current page.south west)},
                      background rectangle/.style={fill=pantone655},
                      show background rectangle]
    \node [anchor=west, align=left, inner sep=0, text=white]
          (lecturenumber) at (\paperwidth / 6, \paperheight * 3 / 4)
          {\Large Lecture \insertlecturenumber};
    \node [inner sep=0, align=left, text=white, node distance=0,
          above left=of lecturenumber, anchor=south west, yshift=2mm]
          {\Large ECE 344: Operating Systems};
    \node (title) [inner sep=0, anchor=west, align=left, text=white,
                   text width=30em]
          at (\paperwidth / 6, \paperheight / 2)
          {{\bfseries \Huge \inserttitle{}}};
    \node [inner sep=0, align=right, text=white, node distance=0,
          below right=of title, anchor=north east, yshift=-1mm]
          {{\footnotesize \ttfamily \insertversion}};
    \node [inner sep=0, text=white, align=left, anchor=west]
          (author) at (\paperwidth / 6, \paperheight / 4)
          {\insertauthor};
    \node [text=white, inner sep=0, align=left, node distance=0,
           below left=of author, anchor=north west, yshift=-2mm]
          {\insertdate};
    \node [align=right, anchor=south east, inner sep=2mm, text=white]
          (license) at (\paperwidth, 0)
          {\footnotesize This  work is licensed under a
           \href{http://creativecommons.org/licenses/by-sa/4.0/}
                {\color{pantone7453} Creative Commons Attribution-ShareAlike 4.0
                 International License}};
    \node [text=white, inner sep=0, align=right, node distance=0,
           above right=of license, anchor=south east, xshift=-2mm]
          {\Large \ccbysa};
  \end{tikzpicture}
}

\tikzset{
  >=Straight Barb[],
  shorten >=1pt,
  initial text=,
}

\lstset{
  basicstyle=\footnotesize\ttfamily,
  language=C,
  escapechar=@,
  commentstyle=\color{black!50},
}


\lecturenumber{18}
\title{Quiz 2 Review}
\version{1.0.0}
\author{Jon Eyolfson}
\date{October 20, 2022}

\begin{document}
  \begin{frame}[plain, noframenumbering]
    \titlepage
  \end{frame}

  \begin{frame}[fragile]
    \frametitle{A Forking Question}

    Consider the following code:

    \begin{lstlisting}
int main() {
  pid_t first = fork();
  pid_t second = fork();
  pid_t third = fork();
  printf("first=%d second=%d third=%d\n", first, second, third);
}
    \end{lstlisting}

    \vspace{2em}

    What is one reasonable set of outputs (assume the initial process is pid 2)?

    \vspace{1em}

    What order are the outputs in?

    \vspace{1em}

    What do the relationships between processes look like?
  \end{frame}

  \begin{frame}[fragile]
    \frametitle{A Threading Question}

    Assume you have a global variable, \lstinline!int i;! and 3 threads.

    \vspace{1em}

    Before any thread executes, \lstinline!i! is initialized to \lstinline!0!

    \vspace{1em}

    Two threads execute \lstinline!i++! and one executes \lstinline!i--!

    \vspace{2em}

    What are all the possible values of \lstinline!i! after all threads execute?
  \end{frame}

  \begin{frame}
    \frametitle{Unix Systems Clone Processes with a Parent/Child Relationship}

    \begin{itemize}
      \item You can only create new processes with \texttt{fork}
      \item After a \texttt{fork} both processes are exactly the same
      \begin{itemize}
        \item except for the value of \texttt{pid} (the child is always 0)
      \end{itemize}
      \item The scheduler decides when to run either process
    \end{itemize}
  \end{frame}

  \begin{frame}
    \frametitle{You're Responsible for Managing Processes}

    The operating system maintains a strict parent/child relationship
    
    \vspace{2em}
    
    You should be able to identify (and prevent) the following:
    \begin{itemize}
      \item Zombie processes
      \item Orphan processes
    \end{itemize}

    \vspace{2em}

    \textit{For quiz 2: the focus would be more on \texttt{wait}}
  \end{frame}
  
  \begin{frame}
    \frametitle{We Explored Basic IPC in an Operating System}

    Some basic IPC includes:
    \begin{itemize}
      \item \texttt{read} and \texttt{write} through file descriptors (could be a regular file)
      \item Redirecting file descriptors for communcation
      \item Signals
    \end{itemize}

    \vspace{2em}

    Signals are like interrupts for user processes

    \hspace{2em} The kernel has to handle all 3 kinds of ``interrupts''

    \vspace{2em}

    \textit{For quiz 2: this isn't covered}
  \end{frame}

  \begin{frame}
    \frametitle{Threads Enable Concurrency}

    We explored threads, and related them to something we already know (processes)
    \begin{itemize}
      \item Threads are lighter weight, and share memory by default
      \item Each process can have multiple threads (but just one at the start)
    \end{itemize}
  \end{frame}

  \begin{frame}
    \frametitle{Both Processes and (Kernel) Threads Enable Parallelization}

    \begin{itemize}
      \item Each process can have multiple (kernel) threads
      \item Most implementations use one-to-one user-to-kernel thread mapping
      \item The operating system has to manage what happens during a fork, or signals
      \item We now have synchronization issues
    \end{itemize}
  \end{frame}

  \begin{frame}
    \frametitle{We Want Critical Sections to Protect Against Data Races}

    We should know what data races are, and how to prevent them:
    \begin{itemize}
      \item Mutex or spinlocks are the most straightforward locks
      \item We need hardware support to implement locks
      \item We need some kernel support for wake up notifications
      \item If we know we have a lot of readers, we should use a read-write lock
    \end{itemize}
  \end{frame}
  
  \begin{frame}
    \frametitle{We Used Semaphores to Ensure Proper Order}

    Previously we ensured mutual exclusion, now we can ensure order

    \begin{itemize}
      \item Semaphores contain an initial value you choose
      \item You can increment the value using post
      \item You can decrement the value using wait (it blocks if the current
            value is 0)
      \item You still need to be prevent data races
    \end{itemize}
  \end{frame}

  \begin{frame}
    \frametitle{We Explored More Advanced Locking}

    We have another tool to ensure order

    \begin{itemize}
      \item Condition variables are clearer for complex condition signaling
      \item Locking granularity matters
      \item You must prevent deadlocks
    \end{itemize}
  \end{frame}

  \begin{frame}
    \frametitle{Final Questions or Concerns?}

    Ashvin and I will be on Discord, we'll announce any issues (hopefully none!)
    
    \vspace{2em}

    Format: 2 true/false, 2 multiple choice, 2 short answer (free form)

    \vspace{2em}

    Expect to take around 47 minutes total:
    
    \hspace{2em} 1 minute for true/false,
    
    \hspace{2em} 10 minutes for each multiple choice,

    \hspace{2em} 10 minutes for the first short answer,
    
    \hspace{2em} and 15 minutes for the final short answer.

    \vspace{2em}

    Good luck!
  \end{frame}
\end{document}
