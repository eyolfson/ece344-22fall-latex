\documentclass[aspectratio=169]{beamer}

\usepackage{ccicons}
\usepackage{fontspec}
\usepackage{listings}
\usepackage{tikz}
\usepackage{svg}

\definecolor{uclablue}{RGB}{39,116,174}
\definecolor{uclagold}{RGB}{255,179,0}

\definecolor{ubcorange}{RGB}{158, 66, 37}

\definecolor{cugold}{RGB}{207, 184, 124}
\definecolor{cudarkgray}{RGB}{86, 90, 92}

\definecolor{solarizedred}{RGB}{220, 50, 47}
\definecolor{solarizedblue}{RGB}{38, 139, 210}
\definecolor{solarizedgreen}{RGB}{133, 153, 0}
\definecolor{solarizedpurple}{RGB}{108, 113, 196}
\definecolor{solarizedmagenta}{RGB}{211, 54, 130}

\definecolor{pantone655}{RGB}{0, 42, 92}
\definecolor{pantone7453}{RGB}{123, 164, 217}
\definecolor{pantone633}{RGB}{0, 139, 176}
\definecolor{pantone7492}{RGB}{218, 229, 205}

\colorlet{primarycolor}{pantone655}
\colorlet{secondarycolor}{pantone7453}


\usetikzlibrary{
  arrows,
  arrows.meta,
  automata,
  backgrounds,
  calc,
  chains,
  decorations.pathreplacing,
  fit,
  intersections,
  matrix,
  overlay-beamer-styles,
  positioning,
  shapes,
  shapes.multipart,
  tikzmark,
}
\usetikzmarklibrary{listings}

\hypersetup{
  colorlinks=true,
  urlcolor=cudarkgray,
}

\setbeamercolor{frametitle}{fg=primarycolor}
\setbeamercolor{structure}{fg=primarycolor}
\setbeamercolor{enumerate item}{fg=black}
\setbeamercolor{itemize item}{fg=black}
\setbeamercolor{itemize subitem}{fg=black}

\setbeamersize{text margin left=26.6mm}
\addtolength{\headsep}{2mm}

\setbeamertemplate{navigation symbols}{}
\setbeamertemplate{headline}{}
\setbeamertemplate{footline}{}
\setbeamertemplate{itemize item}{\color{black}}
\setbeamertemplate{itemize items}[circle]

\setbeamertemplate{footline}{
  \begin{tikzpicture}[remember picture,
                      overlay,
                      shift={(current page.south west)}]
    \node [black!50, inner sep=2mm, anchor=south east]
          at (current page.south east) {\footnotesize \insertframenumber};
  \end{tikzpicture}
}

\setsansfont{Inter}[Scale=MatchLowercase]
\setmonofont{Hack}[Scale=MatchLowercase]

\makeatletter
\newcommand\version[1]{\renewcommand\@version{#1}}
\newcommand\@version{}
\def\insertversion{\@version}

\newcommand\lecturenumber[1]{\renewcommand\@lecturenumber{#1}}
\newcommand\@lecturenumber{}
\def\insertlecturenumber{\@lecturenumber}
\makeatother

\setbeamertemplate{title page}
{
  \begin{tikzpicture}[remember picture,
                      overlay,
                      shift={(current page.south west)},
                      background rectangle/.style={fill=pantone655},
                      show background rectangle]
    \node [anchor=west, align=left, inner sep=0, text=white]
          (lecturenumber) at (\paperwidth / 6, \paperheight * 3 / 4)
          {\Large Lecture \insertlecturenumber};
    \node [inner sep=0, align=left, text=white, node distance=0,
          above left=of lecturenumber, anchor=south west, yshift=2mm]
          {\Large ECE 344: Operating Systems};
    \node (title) [inner sep=0, anchor=west, align=left, text=white,
                   text width=30em]
          at (\paperwidth / 6, \paperheight / 2)
          {{\bfseries \Huge \inserttitle{}}};
    \node [inner sep=0, align=right, text=white, node distance=0,
          below right=of title, anchor=north east, yshift=-1mm]
          {{\footnotesize \ttfamily \insertversion}};
    \node [inner sep=0, text=white, align=left, anchor=west]
          (author) at (\paperwidth / 6, \paperheight / 4)
          {\insertauthor};
    \node [text=white, inner sep=0, align=left, node distance=0,
           below left=of author, anchor=north west, yshift=-2mm]
          {\insertdate};
    \node [align=right, anchor=south east, inner sep=2mm, text=white]
          (license) at (\paperwidth, 0)
          {\footnotesize This  work is licensed under a
           \href{http://creativecommons.org/licenses/by-sa/4.0/}
                {\color{pantone7453} Creative Commons Attribution-ShareAlike 4.0
                 International License}};
    \node [text=white, inner sep=0, align=right, node distance=0,
           above right=of license, anchor=south east, xshift=-2mm]
          {\Large \ccbysa};
  \end{tikzpicture}
}

\tikzset{
  >=Straight Barb[],
  shorten >=1pt,
  initial text=,
}

\lstset{
  basicstyle=\footnotesize\ttfamily,
  language=C,
  escapechar=@,
  commentstyle=\color{black!50},
}


\lecturenumber{12}
\title{Locking}
\version{1.2.0}
\author{Jon Eyolfson}
\date{October 4/5, 2021}

\begin{document}
  \begin{frame}[plain, noframenumbering]
    \titlepage
  \end{frame}

  \begin{frame}
    \frametitle{Monitors Are Built Into Some Languages}

    With object oriented programming, developers wanted something easier to use

    \vspace{2em}

    Could mark a method as monitored, and let the compiler handle locking

    \hspace{2em} An object can only have one thread active in its monitored
    methods

    \vspace{2em}

    It's basically one mutex per object, created for you

    \hspace{2em} The compiler inserts calls to lock and unlock
  \end{frame}

  \begin{frame}[fragile]
    \frametitle{Java's \texttt{synchronized} Keyword is an Example of a Monitor}

    \begin{lstlisting}
public class Account {
  int balance;
  public synchronized void deposit(int amount)  { balance += amount; }
  public synchronized void withdraw(int amount) { balance -= amount; }
}
    \end{lstlisting}

    the compiler transforms to:

    \begin{lstlisting}[escapechar=!]
  public void deposit(int amount) {
    !\structure{lock(this.monitor);}!
    balance += amount;
    !\structure{unlock(this.monitor);}!
  }
  public void withdraw(int amount) {
    !\structure{lock(this.monitor);}!
    balance -= amount;
    !\structure{unlock(this.monitor);}!
  }
    \end{lstlisting}
  \end{frame}

  \begin{frame}[fragile]
    \frametitle{Condition Variables Behave Like Semaphores}

    You can create your own custom queue of threads

    \vspace{2em}

    \begin{lstlisting}
#include <pthread.h>

int pthread_cond_init(pthread_cond_t *cond,
                      const pthread_condattr_t *attr)
int pthread_cond_destroy(pthread_cond_t *cond);
int pthread_cond_signal(pthread_cond_t *cond);
int pthread_cond_broadcast(pthread_cond_t *cond);
int pthread_cond_wait(pthread_cond_t *cond, pthread_mutex_t *mutex);
int pthread_cond_timedwait(pthread_cond_t *cond, pthread_mutex_t *mutex,
                           const struct timespec *abstime);
    \end{lstlisting}

    \vspace{2em}

    The \texttt{wait} functions add this thread to the queue

    \hspace{2em} \texttt{signal} wakes up one thread, \texttt{broadcast} wakes
    up all threads
  \end{frame}

  \begin{frame}
    \frametitle{Condition Variables MUST Be Paired with a Mutex}

    Any calls to \texttt{wait} must already hold the mutex

    \hspace{2em} (\texttt{signal} and \texttt{broadcast} may not)

    \vspace{2em}

    Why? You may think \texttt{wait} needs to add itself to the queue safely (no
    data races)

    \hspace{2em} It needs the mutex as an argument to unlock it before going to
    sleep

    \hspace{4em} The thread will hold the locked \texttt{mutex} before and after
    the call

    \vspace{2em}

    One mutex can also protect multiple condition variables

    \vspace{2em}

    We'll only consider calls to \texttt{wait} and \texttt{signal}
  \end{frame}

  \begin{frame}
    \frametitle{The \texttt{wait} Call Does Not Contain Data Races}

    Understand what \texttt{wait} does (for condition variables!)

    \vspace{2em}

    The thread calling \texttt{wait}:
    \begin{enumerate}
      \item Adds itself to the queue for the condition variable
      \item Unlock the mutex
      \item Gets blocked (it can no longer be scheduled to run)
    \end{enumerate}

    \vspace{2em}

    The thread calling \texttt{wait} needs another thread to call \texttt{signal} or
    \texttt{broadcast},
    
    then if it's selected:

    \begin{enumerate}
      \item Gets unblocked (it can be scheduled to run)
      \item Tries to lock the mutex again, \texttt{wait} returns when it gets it
    \end{enumerate}
  \end{frame}

  \begin{frame}[fragile]
    \frametitle{We Can Use Condition Variables for Our Producer/Consumer}

    \begin{columns}
      \begin{column}{0.5\textwidth}
        \begin{lstlisting}
pthread_mutex_t mutex;
int nfilled;
pthread_cond_t has_filled;
pthread_cond_t has_empty;

void producer() {
  // produce data
  pthread_mutex_lock(&mutex);
  while (nfilled == N) {
    pthread_cond_wait(&has_empty,
                      &mutex);
  }
  // fill a slot
  ++nfilled;
  pthread_cond_signal(&has_filled);
  pthread_mutex_unlock(&mutex);
}
        \end{lstlisting}
      \end{column}
      \begin{column}{0.5\textwidth}
        \begin{lstlisting}
void consumer() {
  pthread_mutex_lock(&mutex);
  while (nfilled == 0) {
    pthread_cond_wait(&has_filled,
                      &mutex);
  }
  // empty a slot
  --nfilled;
  pthread_cond_signal(&has_empty);
  pthread_mutex_unlock(&mutex);
  // consume data
}
        \end{lstlisting}
      \end{column}
    \end{columns}
  \end{frame}

  \begin{frame}[fragile]
    \frametitle{What's the Issue with the Following Code?}

    \begin{lstlisting}
/* Thread 1 */
pthread_mutex_lock(&mutex);
while (!condition) {
  pthread_cond_wait(&cond, &mutex);
}
pthread_mutex_unlock(&mutex);

/* Thread 2 */
condition = true;
pthread_cond_signal(&cond);
    \end{lstlisting}
  \end{frame}

  \begin{frame}[fragile]
    \frametitle{Can We Change the \texttt{while} to an \texttt{if}?}

    \begin{lstlisting}
/* Thread 1 */
pthread_mutex_lock(&mutex);
if (!condition) {
  pthread_cond_wait(&cond, &mutex);
}
// What could happen here?
pthread_mutex_unlock(&mutex);

/* Thread 2 */
pthread_mutex_lock(&mutex);
condition = true;
pthread_mutex_unlock(&mutex);
pthread_cond_signal(&cond);

/* Thread 3 */
pthread_mutex_lock(&mutex);
condition = false;
pthread_mutex_unlock(&mutex);
    \end{lstlisting}
  \end{frame}

  \begin{frame}
    \frametitle{Condition Variables Serve a Similar Purpose as Semaphores}

    You can think of semaphores as a special case of condition variables

    \hspace{2em} They'll go to sleep when the value is 0, when it's greater
    than 0 they wake up

    \vspace{2em}

    You can implement one using the other, however it can get messy

    \vspace{2em}

    For complex conditions condition variables offer much better clarity
  \end{frame}

  \begin{frame}
    \frametitle{Locking Granularity is the Extent of Your Locks}

    You need locks to prevent data races

    \vspace{2em}

    Lock large sections of your program, or divide the locks and
    use smaller sections?

    \hspace{2em} What if you want to parallelize your hash table?

    \vspace{2em}
    
    Things to consider about locks:

    \begin{itemize}
     \item Overhead
      \item Contention
      \item Deadlocks
    \end{itemize}
  \end{frame}

  \begin{frame}
    \frametitle{Locking Overheads}

    \begin{itemize}
      \item Memory allocated
      \item Initialization and destruction time
      \item Time to acquire and release locks
    \end{itemize}

    \vspace{2em}

    The more locks you have, the greater each cost is going to be
  \end{frame}

  \begin{frame}
    \frametitle{You Do NOT Want Deadlocks}

    The more locks you have, the more you have to worry about deadlocks

    \vspace{2em}

    Conditions for deadlocking:

    \begin{enumerate}
      \item Mutual Exclusion (of course for simple locks)
      \item Hold and Wait (you have a lock and try to acquire another)
      \item No Preemption (we can't take simple locks away)
      \item Circular Wait (waiting for a lock held by another process)
    \end{enumerate}
  \end{frame}

  \begin{frame}[containsverbatim]
    \frametitle{A Simple Deadlock Example}

    Consider two processors trying to get two {\it locks}:

    \vspace{2em}

    \begin{columns}
      \column{0.4\textwidth}
        {\bf Thread 1}

        \verb+Get Lock 1+

        \verb+Get Lock 2+

        \verb+Release Lock 2+

        \verb+Release Lock 1+
      \column{0.4\textwidth}
        {\bf Thread 2}

        \verb+Get Lock 2+

        \verb+Get Lock 1+

        \verb+Release Lock 1+

        \verb+Release Lock 2+
    \end{columns}

    \vspace{2em}

    Thread 1 gets Lock 1, then Thread 2 gets Lock 2, now
    they both wait for each other

    \hspace{2em} Deadlock
  \end{frame}

  \begin{frame}[fragile]
    \frametitle{You Can Ensure Order to Prevent Deadlocks}

    \begin{lstlisting}
void f1() {
    locktype_lock(&l1);
    locktype_lock(&l2);
    // protected code
    locktype_unlock(&l2);
    locktype_unlock(&ll);    
}
    \end{lstlisting}

    This code will not deadlock, you can only get {\tt l2} if you have
    {\tt l1}
  \end{frame}

  \begin{frame}[fragile]
    \frametitle{You Could Also Prevent A Deadlock by Using {\tt trylock}}

    Remember, for {\tt pthread} there's {\tt trylock} that returns 0 if it gets
    the lock

    \begin{lstlisting}
void f2() {
    locktype_lock(&l1);
    while (locktype_trylock(&l2) != 0) {
        locktype_unlock(&l1);
        // wait
        locktype_lock(&l1);
    }
    // protected code
    locktype_unlock(&l2);
    locktype_unlock(&ll);    
}
    \end{lstlisting}

    This code will not deadlock, it will give up {\tt l1} if it can't get
    {\tt l2}
  \end{frame}

  \begin{frame}
    \frametitle{We Explored More Advanced Locking}

    We have another tool to ensure order

    \begin{itemize}
      \item Condition variables are clearer for complex condition signaling
      \item Locking granularity matters
      \item You must prevent deadlocks
    \end{itemize}
  \end{frame}
\end{document}
