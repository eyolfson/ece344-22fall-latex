\documentclass[aspectratio=169]{beamer}

\usepackage{ccicons}
\usepackage{fontspec}
\usepackage{listings}
\usepackage{tikz}
\usepackage{svg}

\definecolor{uclablue}{RGB}{39,116,174}
\definecolor{uclagold}{RGB}{255,179,0}

\definecolor{ubcorange}{RGB}{158, 66, 37}

\definecolor{cugold}{RGB}{207, 184, 124}
\definecolor{cudarkgray}{RGB}{86, 90, 92}

\definecolor{solarizedred}{RGB}{220, 50, 47}
\definecolor{solarizedblue}{RGB}{38, 139, 210}
\definecolor{solarizedgreen}{RGB}{133, 153, 0}
\definecolor{solarizedpurple}{RGB}{108, 113, 196}
\definecolor{solarizedmagenta}{RGB}{211, 54, 130}

\definecolor{pantone655}{RGB}{0, 42, 92}
\definecolor{pantone7453}{RGB}{123, 164, 217}
\definecolor{pantone633}{RGB}{0, 139, 176}
\definecolor{pantone7492}{RGB}{218, 229, 205}

\colorlet{primarycolor}{pantone655}
\colorlet{secondarycolor}{pantone7453}


\usetikzlibrary{
  arrows,
  arrows.meta,
  automata,
  backgrounds,
  calc,
  chains,
  decorations.pathreplacing,
  fit,
  intersections,
  matrix,
  overlay-beamer-styles,
  positioning,
  shapes,
  shapes.multipart,
  tikzmark,
}
\usetikzmarklibrary{listings}

\hypersetup{
  colorlinks=true,
  urlcolor=cudarkgray,
}

\setbeamercolor{frametitle}{fg=primarycolor}
\setbeamercolor{structure}{fg=primarycolor}
\setbeamercolor{enumerate item}{fg=black}
\setbeamercolor{itemize item}{fg=black}
\setbeamercolor{itemize subitem}{fg=black}

\setbeamersize{text margin left=26.6mm}
\addtolength{\headsep}{2mm}

\setbeamertemplate{navigation symbols}{}
\setbeamertemplate{headline}{}
\setbeamertemplate{footline}{}
\setbeamertemplate{itemize item}{\color{black}}
\setbeamertemplate{itemize items}[circle]

\setbeamertemplate{footline}{
  \begin{tikzpicture}[remember picture,
                      overlay,
                      shift={(current page.south west)}]
    \node [black!50, inner sep=2mm, anchor=south east]
          at (current page.south east) {\footnotesize \insertframenumber};
  \end{tikzpicture}
}

\setsansfont{Inter}[Scale=MatchLowercase]
\setmonofont{Hack}[Scale=MatchLowercase]

\makeatletter
\newcommand\version[1]{\renewcommand\@version{#1}}
\newcommand\@version{}
\def\insertversion{\@version}

\newcommand\lecturenumber[1]{\renewcommand\@lecturenumber{#1}}
\newcommand\@lecturenumber{}
\def\insertlecturenumber{\@lecturenumber}
\makeatother

\setbeamertemplate{title page}
{
  \begin{tikzpicture}[remember picture,
                      overlay,
                      shift={(current page.south west)},
                      background rectangle/.style={fill=pantone655},
                      show background rectangle]
    \node [anchor=west, align=left, inner sep=0, text=white]
          (lecturenumber) at (\paperwidth / 6, \paperheight * 3 / 4)
          {\Large Lecture \insertlecturenumber};
    \node [inner sep=0, align=left, text=white, node distance=0,
          above left=of lecturenumber, anchor=south west, yshift=2mm]
          {\Large ECE 344: Operating Systems};
    \node (title) [inner sep=0, anchor=west, align=left, text=white,
                   text width=30em]
          at (\paperwidth / 6, \paperheight / 2)
          {{\bfseries \Huge \inserttitle{}}};
    \node [inner sep=0, align=right, text=white, node distance=0,
          below right=of title, anchor=north east, yshift=-1mm]
          {{\footnotesize \ttfamily \insertversion}};
    \node [inner sep=0, text=white, align=left, anchor=west]
          (author) at (\paperwidth / 6, \paperheight / 4)
          {\insertauthor};
    \node [text=white, inner sep=0, align=left, node distance=0,
           below left=of author, anchor=north west, yshift=-2mm]
          {\insertdate};
    \node [align=right, anchor=south east, inner sep=2mm, text=white]
          (license) at (\paperwidth, 0)
          {\footnotesize This  work is licensed under a
           \href{http://creativecommons.org/licenses/by-sa/4.0/}
                {\color{pantone7453} Creative Commons Attribution-ShareAlike 4.0
                 International License}};
    \node [text=white, inner sep=0, align=right, node distance=0,
           above right=of license, anchor=south east, xshift=-2mm]
          {\Large \ccbysa};
  \end{tikzpicture}
}

\tikzset{
  >=Straight Barb[],
  shorten >=1pt,
  initial text=,
}

\lstset{
  basicstyle=\footnotesize\ttfamily,
  language=C,
  escapechar=@,
  commentstyle=\color{black!50},
}


\lecturenumber{6}
\title{Basic IPC}
\version{1.0.0}
\author{Jon Eyolfson}
\date{September 20/21, 2022}

\begin{document}
  \begin{frame}[plain, noframenumbering]
    \titlepage
  \end{frame}

  \begin{frame}
    \frametitle{IPC is Transferring Bytes Between Two or More Processes}

    Reading and writing files is a form of IPC

    \vspace{2em}

    The read and write system calls allow any bytes
  \end{frame}

  \begin{frame}
    \frametitle{A Simple Process Could Write Everything It Reads}

    See: \texttt{lecture-06/read-write-example.c}

    \vspace{2em}

    We \texttt{read} from standard in, and \texttt{write} to standard out

    \hspace{2em} Does this remind you of any program you've seen before?

    \vspace{2em}

    If we run it in our terminal without arguments, it'll wait for input

    \hspace{2em} Press Ctrl+D when you're done to send end-of-file (EOF)
  \end{frame}

  \begin{frame}
    \frametitle{\texttt{read} Just Reads Data from a File Descriptor}

    See: \texttt{man 2 read}

    \vspace{2em}

    There's no EOF character, \texttt{read} just returns 0 bytes read

    \hspace{2em} The kernel returns 0 on a closed file descriptor

    \vspace{2em}

    We need to check for errors!

    \hspace{2em} Save \texttt{errno} if you're using another function that may
                 set it
  \end{frame}

  \begin{frame}
    \frametitle{\texttt{write} Just Writes Data to a File Descriptor}

    See: \texttt{man 2 write}

    \vspace{2em}

    It returns the number of bytes written, you can't assume it's always
    successful

    \hspace{2em} Save \texttt{errno} if you're using another function that may
                 set it

    \vspace{2em}

    Both ends of the read and write have a corresponding write and read

    \hspace{2em} This makes two communication channels with command line programs
  \end{frame}

  \begin{frame}
    \frametitle{The Standard File Descriptors Are Powerful}

    We could close standard input (freeing file descriptor 0) and open a file
    instead

    \hspace{2em} Linux uses the lowest available file descriptor for new ones

    \vspace{2em}

    See: \texttt{lecture-06/open-example.c} and \texttt{man 2 open}
    
    \vspace{2em}

    Without changing the core code, it now works with multiple input types

    \vspace{2em} You could type, or use a file
  \end{frame}

  \begin{frame}
    \frametitle{Your Shell Will Let You Redirect Standard File Descriptors}

    Instead of running \texttt{./open-example open-example.c} we could run:

    \hspace{2em} \texttt{./open-example < open-example.c}

    \vspace{2em}

    Your shell will do the \texttt{open} for you and replace the standard input

    \hspace{2em} We didn't actually have to write that!

    \vspace{2em}

    You could also redirect across multiple processes

    \hspace{2em} \texttt{cat open-example.c | ./open-example} 
  \end{frame}

  \begin{frame}
    \frametitle{Signals are a Form of IPC that Interrupts}

    You could also press Ctrl+C to stop \texttt{./open-example}

    \hspace{2em} This interrupts your programs execution and exits early

    \vspace{2em}

    Kernel sends a number to your program indicating the type of signal
    
    \hspace{2em} Kernel default handlers either ignore the signal or terminate
    your process

    \vspace{2em}

    Ctrl+C sends \texttt{SIGINT} (interrupt from keyboard)

    \vspace{2em}

    If the default handler occurs the exit code will be 128 + signal number
  \end{frame}

  \begin{frame}
    \frametitle{You Can Set Your Own Signal Handlers with \texttt{sigaction}}

    See: \texttt{lecture-06/signal-example.c} and \texttt{man 2 sigaction}

    \vspace{2em}

    You just declare a function that doesn't return a value, and has an \texttt{int} argument

    \hspace{2em} The integer is the signal number

    \vspace{2em}

    Some numbers are non standard, here a few from Linux x86-64:
    \begin{itemize}
      \item 2: \texttt{SIGINT} (interrupt from keyboard)
      \item 9: \texttt{SIGKILL} (terminate immediately)
      \item 11: \texttt{SIGSEGV} (memory access violation)
      \item 15: \texttt{SIGTERM} (terminate)
    \end{itemize}
  \end{frame}

  \begin{frame}
    \frametitle{A Signal Pauses Your Process and Runs the Signal Handler}

    Your process can be interrupted at any point in execution

    \hspace{2em} Your process resumes after the signal handler finishes

    \vspace{2em}

    This is an example of concurrency, your process switches execution

    \hspace{2em} You have to be careful what you write here

    \vspace{2em}

    Run \texttt{./signal-example} and press Ctrl+C
  \end{frame}

  \begin{frame}
    \frametitle{You Need to Account for Interrupted System Calls}

    You should see:

    \hspace{2em} \texttt{Ignoring signal 2}

    \hspace{2em} \texttt{read: Interrupted system call}

    \vspace{2em}

    We can rewrite it to retry interrupted system calls

    \hspace{2em} See: \texttt{lecture-06/signal-example-2.c}

    \vspace{2em}

    Now the program continues when we press Ctrl+C
  \end{frame}

  \begin{frame}
    \frametitle{You Can Send Signals to Processes with Their PID}

    You can use the command: \texttt{kill}

    \hspace{2em} It is also a system call, taking a \texttt{pid} and signal number

    \vspace{2em}

    Find a processes' ID with \texttt{pidof}, e.g. \texttt{pidof ./signal-example-2}

    \vspace{2em}

    After use \texttt{kill <pid>}, which by default sends \texttt{SIGTERM}

    \vspace{2em}

    Use \texttt{kill -9 <pid>} to tell the kernel to terminate the process

    \hspace{2em} Process won't terminate if it's in uninterruptible sleep
  \end{frame}

  \begin{frame}
    \frametitle{Most Operations Are Non-Blocking}

    A non-blocking call returns immediately, and you check if something occurs

    \vspace{2em}

    To turn \texttt{wait} into a non-blocking call, use \texttt{waitpid} with
    \texttt{WNOHANG} in \texttt{options}

    \vspace{2em}

    To react to changes to a non-blocking call, we can either use a
    \textit{poll} or \textit{interrupt}
  \end{frame}

  \begin{frame}
    \frametitle{Polling Continuously Calls the Function and Checks for Changes}

    See: \texttt{lecture-06/signal-poll-example.c}

    \vspace{2em}

    We call \texttt{waitpid} over and over until the child exits

    \hspace{2em} Note: some hardware behaves like this,
    
    \hspace{5em} the kernel may have to check for changes

    \vspace{2em}

    What's the drawback of this approach?

  \end{frame}

  \begin{frame}
    \frametitle{An Interrupt Instead Occurs Right After the Change}

    See: \texttt{lecture-06/signal-interrupt-example.c}

    \vspace{2em}

    Instead of calling \texttt{wait} or \texttt{waitpid} from \texttt{main},
    we can do it in the interrupt handler

    \hspace{2em} The kernel sends the \texttt{SIGCHLD} whenever one of its
                 children exit

    \vspace{2em}

    This idea also applies to the kernel, hardware can generate interrupts
  \end{frame}

  \begin{frame}
    \frametitle{Interrupt Handlers Run to Completion}

    See: \texttt{lecture-06/signal-close-example.c}

    \vspace{2em}

    An interrupt may occur while an interrupt handler is already running

    \vspace{2em}

    All interrupt handler code must be reentrant

    \hspace{2em} You need to be able to pause execution,
    
    \hspace{2em} execute another call (to the same function),
    
    \hspace{2em} and resume execution
  \end{frame}

  \begin{frame}
    \frametitle{On a CPU, There's 3 Kinds of ``Interrupts''}

    \textit{Interrupt}

    \hspace{2em} Triggered by external hardware,
    
    \hspace{2em} handled by the kernel (needs to respond quickly)

    \vspace{2em}

    \textit{Trap}

    \hspace{2em} Triggered by users (a system call is a trap),
    
    \hspace{2em} handled by the kernel (calling process suspended)

    \vspace{2em}

    \textit{Exception}

    \hspace{2em} Triggered by abnormal control flow (divide by zero, illegal memory access),

    \hspace{2em} default handler is the kernel (calling process suspended),

    \hspace{2em} the process can optionally handle these themselves
  \end{frame}

  \begin{frame}
    \frametitle{We Explored Basic IPC in an Operating System}

    Some basic IPC includes:
    \begin{itemize}
      \item \texttt{read} and \texttt{write} through file descriptors (could be a regular file)
      \item Redirecting file descriptors for communcation
      \item Signals
    \end{itemize}

    \vspace{2em}

    Signals are like interrupts for user processes

    \hspace{2em} The kernel has to handle all 3 kinds of ``interrupts''
  \end{frame}
\end{document}
